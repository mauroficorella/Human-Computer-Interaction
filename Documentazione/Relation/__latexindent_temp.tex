\documentclass[12pt, a4paper]{article}
\usepackage[a4paper,
left=15mm,
right=15mm,
top=15mm,
bottom = 5mm]{geometry}
\usepackage{amsmath}
\usepackage{graphicx}
\usepackage{algorithm}
\usepackage{algorithmic}
\usepackage{multicol}
\usepackage{amsthm}
\usepackage{bm}
\usepackage{fancyhdr}

\newcommand\tab[1][1cm]{\hspace*{#1}}
\renewcommand{\labelitemii}{$\star$}

\begin{document}


\begin{titlepage}
	\begin{center}
		\vspace*{1cm}
		
		\Huge{Tesina\\Human Computer Interaction}
		\vspace{1.5cm}
		\Huge
		\textbf{\\CiakTime}
		\vspace{1.5cm}
		
		\Large
		Authors:\\
		\textbf{Mauro Ficorella 1941639}\\
		\textbf{Martina Turbessi 1944497}\\
		\textbf{Valentina Sisti 1952657}\\
		\vspace{0.5cm}
		
		\vfill
		
		\includegraphics[width=0.4\textwidth]{Images/Logo.jpg}
		
		\vfill
		
		\vspace{0.8cm}
		
		\Large
		Sapienza\\
		July 2021
	\end{center}
\end{titlepage}

\tableofcontents{}

% INTRODUZIONE --------------------------------------------------------------------

\newpage

\section{Introduction}

The idea behind CiakTime comes from the fact that nowadays there are a lot of movies out 
both in theatres and in streaming platforms, so that cinema lovers can satisfy their
needs to stay updated with the latest movies and keep track of them. 
Also they may want to know on which streaming platforms they can found the movies.
Finally they could have the necessity to interact with other cinema lovers about their 
favourite movies or share their opinion about the movie with the community through reviews.
\\\\
For these reasons, our app offers a lot of functionalities. 
The user has the possibility to keep track of already watched movies, movies to watch and favourite movies;
moreover he can search for movies by title, also filtering results, search for actors and movie directors, 
look for upcoming movies and popular movies and actors.
Regarding the movies, he can read information about plot, cast, year of release, duration, genre,
movie director and streaming platform on which the movie is available; in addition, he can 
review and rate movies, comment and like reviews made by other users.
Finally, regarding movie directors and actors, the user can read their biography and take a look to their
filmography.
\\\\
In order to involve as much users as possible, we decided to make our app available for both iOS and Android devices.





% CAPITOLO 1 -------------------------------------------------------------------------

\newpage

\section{Requirement analysis}

\subsection{Competitor analysis}
We found two main competitors for our application: IMDb and Cinemaniac.
\paragraph{IMDb}\mbox{}\\\\
\includegraphics[width=0.4\textwidth]{Images/IMDb.png}\\
IMDb is the world's most popular and authoritative source for movie, TV, and celebrity information. This app has a 
huge fanbase and a limitless cinema database.
On this app the user can watch trailers, get showtimes, and buy tickets for upcoming films. He can rate and review shows he has seen 
and track what he wants to watch using his Watchlist, and he can also get suggestions regarding movies based on it.\\
However, we have identified few weaknesses, such as the impossibility to exchange opinions between users, to keep track
of already watched movies and to save favourite movies in a list; it is also not very intuitive to retrieve movies specific
information due to the high number of functionality offered by the application.

\paragraph{Cinemaniac:}\mbox{}\\\\
\includegraphics[width=0.4\textwidth]{Images/Cinemaniac.png}\\
Cinemaniac is an app on which the user can search for a movie and add it to the “Movies to watch”, “Watched movies” or favourite list.
He can see all the relevant details for any movie and he can leave his own personal grade.
The user can find suggestions on the most popular and top rated movies. 
Moreover, he can find a specific list relative to currently projected movies and upcoming titles.
Also here we have identified some weaknesses, like the fact that the interface is not so user friendly, there is no user interaction,
there are no information about streaming platforms; moreover the search about movies is not so intuitive and there are 
in-app purchases required to remove a

% CAPITOLO 2 -------------------------------------------------------------------------

\newpage

\section*{NOME}

% CAPITOLO 3 -------------------------------------------------------------------------

\newpage

\section*{NOME}

% CAPITOLO 4 -------------------------------------------------------------------------

\newpage

\section*{NOME}

% CAPITOLO 5 -------------------------------------------------------------------------

\newpage

\section*{NOME}

% CAPITOLO 6 -------------------------------------------------------------------------

\newpage

\section*{NOME}

% CAPITOLO 7 -------------------------------------------------------------------------

\newpage

\section*{NOME}

% CAPITOLO 8 -------------------------------------------------------------------------

\newpage

\section*{NOME}

\end{document}

% COSE UTILI --------------------------------------------------------------------------

%\section*{NOME}
%\subsection*{1.1}
%\setlength{\intextsep}{0pt} --> elimina lo spazio
%\vspace{-3mm}
%\hspace*{0cm}

% Font -------------------------------------

%GRASSETTO: \textbf

% Simboli ---------------------------------

%$\leftarrow$

% Elenco puntato ----------------------

%\begin{itemize}
%\setlength\itemsep{0.01em}
%\item 1
%\item 2
%\end{itemize}

% Graffa grande -----------------------

%\[  
%    \left\{ 
%    \begin{array}{ll} 
%      \mbox{1}
%      \mbox{2}
%    \end{array}
%    \mbox{riga al lato}
%   \right. 
%\]

% Multicolonne --------------------------

% \begin{multicols}{2}
% \columnbreak
% \end{multicols}

% Algoritmi -------------------------------

%\renewcommand{\thealgorithm}{1.\arabic{algorithm}}
%\setcounter{algorithm}{0}
%\begin{algorithm}
%\footnotesize
%\caption{Nome}
%\textbf{Input:} \\
%\textbf{Output:} 

%\begin{algorithmic}[1]
%\STATE 
%\FOR{ = 0 \TO i = n} ---- \ENDFOR
%\IF{} ---- \ELSIF{} ---- \ENDIF
%\RETURN 
%\end{algorithmic}
%\end{algorithm}

% Minipage ------------------------------

%\begin{minipage}[t]{0.5\textwidth}

% queste 3 righe vanno attaccate
%\end{minipage}
%\hspace{0.02\linewidth}
%\begin{minipage}[t]{0.47\textwidth} 

%\begin{minipage}[t]{0.3\textwidth} 
%\end{minipage}

% Proof --------------------------------
%\begin{proof}[\textbf{per cambiare nome}]
%\end{proof}

